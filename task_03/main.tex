\documentclass[a4paper,14pt]{extarticle}
\usepackage[T2A]{fontenc}
\usepackage[utf8]{inputenc}
\usepackage[russian]{babel}
\usepackage[left=3cm,right=2cm,
    top=2cm,bottom=2cm,bindingoffset=0cm]{geometry}
\usepackage{alltt}
\usepackage{listings}
\usepackage{setspace}
\usepackage{enumitem}

\title{Проектирование информационных систем  \\
Домашнее задание № 3}
\author{ФИО: \underline{Леонов В.В.}\\TG:\underline{@leery\underline{ }corsair}\\Группа \underline{М23--524}}
\date{18 октября 2023}

\begin{document}
\maketitle

\tableofcontents
\newpage
\begin{spacing}{1.5}

\section{Постановка задачи}

\hspace{\parindent} Перед вами на экране лицо виртуального персонажа с богатой мимикой, способное выражать различные мимические комплексы на основе FACS: осуществлять движения мышц в области лба, бровей, глаз, рта, губ, подбородка, носогубных складок, щёк. Персонаж может поддерживать беседу и выражать различные эмоции в ответ на реплики пользователя. Мимика, интонация и реплики персонажа определяются моделью или человеком (API для управления предоставляется). 

Необходимо разработать ТЗ к клиентской части приложения (максимально полно).

\section{Требования к клиентской части}

\subsection{Функциональные требования}

\hspace{\parindent} Функциональные требования являются важной частью спецификации приложения и определяют, какие функции и возможности должны быть реализованы в приложении. Деление требований на пользовательские и требования домена (FACS --- Facial Action Coding System) может помочь структурировать их для лучшего понимания и управления.

Пользовательские требования описывают, как пользователи в качестве отдельных сущностей будут взаимодействовать с приложением и какие функции будут доступны:
\begin{enumerate}[label*=\arabic*.]
    \item Регистрация и аутентификация (многофакторная) пользователей.
    \item Вход и выход из системы.
    \item Создание и редактирование профилей пользователей.
    \item Взаимодействие с другими пользователями (обмен сообщениями).
    \item Управление настройками аккаунта.
    \item Конфигурирование параметров уведомлений и оповещений.
\end{enumerate}

Функциональные требования, связанные с FACS-подсистемой:
\begin{enumerate}[label*=\arabic*.]    
    \item С точки зрения персонажа и его кастомизации:
        \begin{enumerate}[label*=\arabic*.]
            \item Наличие различных анимаций движения мышц в области лба, бровей, глаз, рта, губ, подбородка, носогубных складок, щёк персонажа.
            \item Наличие редактора виртуального персонажа (части лица, прическа, голос, раса/человекоподобные существа/животные, визуальный стиль: реализм, мультипликация, аниме и т.д.) и фонового изображения.
            \item Возможность импорта/экспорта конфигурационных файлов.
            \item Поддержка множества персонажей.
        \end{enumerate}
    \item С точки зрения действий персонажа:
        \begin{enumerate}[label*=\arabic*.]
            \item Наличие различных сцериев взаимодействия: повествование за заданную тему, диалог и игровой формат.
            \item Наличие персонализированного психологического образа виртуального персонажа, на основе которого строятся ответные реакции пользователю.
            \item Возможность построения ответов с помощью преднастроенной системы, так с дополнительным использованием сети Интернет.
            \item Возможность построения ответов на основе контекста.
            \item Возможность построения повторного альтернативного ответа.
            \item Наличие фильтрации шокирующего и 18+ контента, а также опции детского режима.
            \item Синхронизация лицевых анимаций и реплик.
        \end{enumerate}
    \item С точки зрения взаимодействия персонажа и пользователя:
        \begin{enumerate}[label*=\arabic*.]
            \item Поддержка взаимодействия с помощью жестов.
            \item Поддержка взаимодействия с помощью голоса.
            \item Поддержка взаимодействия с помощью текста.
            \item Поддержка взаимодействия на различных языках.
            \item Поддержка интеграции со сторонними сервисами с помощью API.
            \item Поддержка множественных диалогов с сохранением истории запросов/ответов.
        \end{enumerate}
\end{enumerate}

\subsubsection{Организация потоков данных}

\hspace{\parindent} Источником первичных входных данных для системы служат действия пользователя в настольном или мобильном приложении, далее они преобразуются и отправляются на сервер в виде HTTP-запросов по сети Интернет.

Выходные данные из системы выводятся пользователю на экране настольного или мобильного приложения, которое получает данные от сервера по сети Интернет в результате HTTP-запросов и преобразует их для отображения пользователю.

\clearpage

\subsection{Требования к интерфейсу}
\hspace{\parindent} Требования к интерфейсу клиентского приложения включают в себя спецификации и ожидания по дизайну, внешнему виду и поведению пользовательского интерфейса (UI). Эти требования помогают обеспечить удовлетворение потребностей пользователей, создать приятное взаимодействие и улучшить общий опыт: 
\begin{enumerate}[label*=\arabic*.]
        \item Поддержка мультиязычности интерфейса, в том числе поддержка разных форматов дат, валют и других локальных особенностей.
        \item Поддержка персонализации: наличие как светлой, так и темной темы, а также режима цветовой слепоты; добавление, удаление и перемещение виджетов на экране.
        \item Соответствие единству стилистической схемы, шрифтов, иконок и досок настроения.
        \item Наличие плавных анимаций для компонентов интерфейса и надежная обратная связь при взаимодействии с пользователем.
        \item Адаптивность под различные платформы (Desktop, Mobile) с корректным масштабированием.
        \item Поддержка широкоформатных устройств (вплоть до 32:9), а также устройств с повышенным разрешением (вплоть до 8К).
        \item Интуитивная система навигации со всплывающими подсказками.
\end{enumerate}

\clearpage

\subsection{Системные требования}

\subsubsection{Общесистемные требования}
\hspace{\parindent} Разрабатываемое клиентское приложение должно быть полным по следующим критериям:
\begin{enumerate}[label*=\arabic*.]
    \item Возможность масштабироваться горизонтально и вертикально, что позволит увеличивать производительность с ростом нагрузки.
    \item Устойчивость к высоким нагрузкам --- обработка большого количества запросов и операций без серьезных сбоев или деградации производительности.
    \item Кэширование --- применение кеширования данных на стороне клиента, чтобы уменьшить нагрузку на сервер и сократить время отклика.
    \item Асинхронные операции и многозадачность для эффективной обработки множества запросов одновременно.
    \item Поддержка всех основных операционных систем: Windows, Linux, MacOS, Android и IOS.
    \item Наличие режима функционирования с потреблением ограниченного числа ресурсов, а также энергоэффективного режима для мобильных устройств.
\end{enumerate}

\subsubsection{Контроль входной информации}
\hspace{\parindent} Контроль входной информации --- это процесс проверки, фильтрации и обработки данных, которые поступают от пользователя или других источников на стороне клиента перед их передачей на сервер. Это важный аспект безопасности и надежности клиент-серверных приложений. Необходимо обеспечить следующие меры:
\begin{enumerate}[label*=\arabic*.]
    \item Валидация --- верификация данных, вводимых пользователями, на соответствие ожидаемым форматам и значениям.
    \item Санитизация данных --- фильтрация входных данных, чтобы удалить или экранировать потенциально опасные символы/скрипты.
    \item Обработка ошибок --- обеспечение обработки ошибок и некорректных данных, чтобы предотвратить сбои приложения из-за неправильных данных.
    \item Защита от межсайтовой подделки запросов --- реализация токенов CSRF, чтобы предотвратить отправку злоумышленниками фальшивых запросов от имени пользователя.
    \item Все данные на сервер передаются в зашифрованном виде. 
\end{enumerate}

\subsubsection{Контроль выходной информации}
\hspace{\parindent} Контроль выходной информации важен для обеспечения безопасности и конфиденциальности данных, которые передаются пользователям или другим системам. Следует реализовать следующие методы и практики контроля выходной информации:
\begin{enumerate}
    \item Экранирование данных --- перед выводом данных на экран или передачей данных на клиентскую сторону, необходимо удостовериться, что все потенциально опасные символы экранированы. Это предотвратит атаки типа XSS.
    \item Защита от атак CSRF --- при передаче данных на клиентскую сторону, необходимо удостовериться, что выполняется защита от атак CSRF, чтобы предотвратить изменение данных без согласия пользователя.
    \item Контроль доступа к API --- при взаимодействии с внешними API, необходимо удостовериться, что доступ к API ограничен и проверяется на наличие соответствующих разрешений.
    \item Управление кешированием --- при кешировании данных на стороне клиента, необходимо удостовериться, что данные в кеше обновляются при изменении на сервере и не сохраняют конфиденциальную информацию.
    \item Логирование безопасности --- ведение журнала событий безопасности, чтобы отслеживать все события, связанные с выходной информацией, и быстро реагировать на возможные инциденты.
\end{enumerate}

\end{spacing}
\end{document}